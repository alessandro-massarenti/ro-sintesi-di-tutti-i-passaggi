\section{Modellazione}

\subsection{Modelli notevoli}

\begin{multicols}{2}

\includegraphics[width=\linewidth]{img/copertura-costo-minimo.png}
\includegraphics[width=\linewidth]{img/mix-ottimo.png}
\includegraphics[width=\linewidth]{img/trasporti.png}
\includegraphics[width=\linewidth]{img/multiperiodale.png}

\end{multicols}
\subsection{Passaggi}
\begin{enumerate}
    \item Individuare le variabili decisionali;
    \item definire la funzione obiettivo;
    \item iniziare a stendere i vincoli non nella parte dei "tenendo conto che:";
    \item iniziare a stendere almeno un vincolo dell'elenco puntato che propone il professore\pro{Almeno uno o due di questi vincoli sono comodi per scrivere il resto del modello};
    \item Scrivere il resto dei vincoli;
    \item Scrivere i domini.
\end{enumerate}

\subsubsection{Individuazione delle variabili decisionali}
Queste variabili sono solitamente a uno o due indici\gab{Spesso le variabii a due indici fanno danni, meglio mettere più variabili ad un indice se possibile.}.
Diventa invece preferibile mettere a due indice quando è evidente dal problema che servono entrambe due colonne di una tabella\footnote{Nel problema dei trasporti solitamente usiamo due indici.}.

Un'altra cosa a cui fare attenzione è la scrittura dei vincoli in cui un \textit{"Indipendentemente da"} ci fa capire che probabilmente avremo bisogno di una variabile ad un indice solo e non due.

\subsubsection{Definizione della funzione obiettivo}
La funzione obiettivo \textbf{DEVE} essere lineare\footnote{Se non è lineare, allora non è un problema di programmazione lineare. (detto più brevemente anche problema di PL).}.

É importante fare molta attenzione a non moltiplicare due variabili, anche se una delle due variabili è binaria.

\subsubsection{Vincoli}

Anche i vincoli \textbf{DEVONO} essere lineari.

\paragraph{I vincoli logici} Se ho vincoli logici, ad esempio "posso decidere se fare X cosa" devo inserire una variabile binaria. Per questi vincoli devo:
\begin{itemize}
    \item definirli nel dominio delle varibili binarie;
    \item attivare\footnote{Linearizzare un modello evitando i termini quadratici.} la variabile utilizzando il \textit{Big-M}
\end{itemize} 

\paragraph{Costi fissi} In questo caso va aggiunto un termine moltiplicativo a tutte le variabili decisionali legate per quantità.
\paragraph{Almeno tot}\footnote{Almeno un treno tra gli $y$ treni deve partire $\sum_{i=1}^{n}y_i >= 1$ } Se il vincolo è logico uso $y_1+y_2+y_n >= \text{Il valore del vincolo}$ con $y$ variabile binaria legata a \textit{Big-M}.

Se il vincolo è di quantità uso: $x_1+x_2+x_n >= \text{Il valore del vincolo}$ con $x$ variabile di quantità.

\paragraph{Al massimo tot} Se il vincolo è logico uso $y_1+y_2+y_n <= \text{Il valore del vincolo}$ con $y$ variabile binaria legata a \textit{Big-M}.

Se il vincolo è di quantità uso $x_1+x_2+x_n <= \text{Il valore del vincolo}$ con $x$ variabile di quantità.

\subsubsection{Consigli ulteriori di Gabriel}
\begin{itemize}
    \item 
    Se si sta scrivendo un vincolo e non si capisce bene come scriverlo, probabilmente può essere scritto più semplicemente, anche ripensando alle variabili che lo compongono;
    \item risemplificare e considerare i punti detti è molto utile.
\end{itemize}


\subsubsection{Consigli di Ale}

\begin{itemize}
    \item Rileggere un paio di volte il problema prima di scrivere qualsiasi cosa;
    \item quando si scrive un vincolo cercare di cancellare a matita la parte che lo descrive per evitare di ripassarci sopra prima della lettura finale;
    \item rileggere tutto alla fine per vedere se si è coperto tutto e per controllare di non aver aggiunto situazioni non lineari involontariamente.
\end{itemize}



